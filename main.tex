\documentclass[acmsmall,screen,review,anonymous,nonacm]{acmart}

\setcitestyle{nosort}
\citestyle{acmauthoryear}

\usepackage[outputdir=build]{minted}
% shorthand for inline code
\newcommand{\hs}[1]{\mintinline{haskell}{#1}}

\usepackage{cleveref}

\title{Async HasChor: Library-Level Fully Asynchronous Choreographic Programming}

\begin{abstract}
Choreographic programming is a paradigm that models a distributed system as a unified program, which is then endpoint projected into individual programs for each node.
%
Thanks to its global point of view, communications among nodes are manifest and guaranteed to match on both ends, resulting in a deadlock-free-by-design construction.
%
However, existing choreographic programming models lack sufficient support for asynchronous communication, as they block both the sender and receiver or, at best, the receiver.
%
This limitation prevents expressing useful protocols that take advantage of concurrency or handle message delays or loesses.

In this paper, we introduce a new choreographic programming model that supports \emph{fully} asynchronous communication, and implement it as Async HasChor, a library in Haskell.
%
The key idea is to make each communication return a \emph{future} and execute sending and receiving asynchronously.
%
One chanllenge of this approach is that messages may arrive out of order, making it difficult to associate them with variables in the program.
%
To address this, we tag each message with a sequence number and use it to connect each sending and receiving.
%
And we extend endpoint projection to automatically generate these sequence numbers consistently across all nodes.
\end{abstract}

\begin{document}

\maketitle
\section{Introduction}

Choreographic programming is a paradigm for programming distributed systems that run on multiple nodes.
%
In choreographic programming, the programmer writes one, unified program --- called a \emph{choreography} --- which defines the entire behavior of a distributed system.
%
The choreography is then \emph{endpoint projected} to individual programs for each node.
%
For example, the following choreography defines a distributed data processing pipeline, where a piece of data is passed around among nodes Alice, Bob, and Carol, with each applying their respective processing function \texttt{f}, \texttt{g}, and \texttt{h} to the data:~\footnote{The code is written in pseudo Async HasChor, the exact syntax and semantics of each operator will be detailed in \Cref{sec:tour}.}

\begin{figure}[h]
\centering
\begin{minipage}{0.4\textwidth}
\begin{minted}{haskell}
pipeline = do
  x <- locally @Alice getData
  y <- comm @Alice @Bob   (f x)
  z <- comm @Bob   @Carol (g y)
  w <- comm @Carol @Alice (h z)
  locally @Alice (showResult w)
\end{minted}
\end{minipage}
\end{figure}

\noindent Endpoint projection will generate the following three individual programs for Alice, Bob, and Carol:

\begin{figure}[h]
\hspace{1cm}
\begin{minipage}[t]{0.3\textwidth}
\begin{minted}{haskell}
alice = do
  x <- getData
  send @Bob (f x)
  w <- recv @Carol
  showResult w
\end{minted}
\end{minipage}
\begin{minipage}[t]{0.3\textwidth}
\begin{minted}{haskell}
bob = do
  y <- recv @Alice
  send @Carol (g y)
\end{minted}
\end{minipage}
\begin{minipage}[t]{0.3\textwidth}
\begin{minted}{haskell}
carol = do
  z <- recv @Bob
  send @Alice (h z)
\end{minted}
\end{minipage}
\end{figure}

Thanks to its global point of view, choreography makes communication among nodes manifest, and they are guaranteed to match with each other, leading to a deadlock-free-by-design construction.
%
In the last decade, several choreographic programming languages have been proposed.
%
More rencently, \emph{library-level} choreographic programming, which embeds choreographic programming in a host language and thus provides a lightweight way to bring choreographic programming to existing languages, have begun to emerge.

Despite these developments, existing choreographic programming languages often assume an model where the underlying network is lossless and synchronous.
%
This assumption (1) makes it impossible to program against message loss or delay, and (2) limits the amount of concurrency as both sender and receiver will be blocked by an instance of communication.

To address these issues, we propose a new choreographic programming model that supports asynchronous communication.
%
Our model incorporates \emph{futures}---a placeholder for a value that will be available in the future---into choreographic programming.
%
We make each communication return a future to represent a yet-to-arrive message.
%
A node can wait on a future, and if the corresponding message has arrived, the carried value will be immediately available for use;
%
if the message has not arrived, the node's execution will be blocked.
%
Futures allow a node to perform a series of communication, and all of them will be executed in parallel, a technique known as
Even better, a node can perform other tasks while futures are being resolved in the background:
%
it can perform more communication, and all of the message passing will be executed in parallel, reducing network latency;
%
it can also wait on
%
With the new future-based communication, prorammers can now express distributed protocols that were impossible before.

We implment our language Async HasChor as an extention to the library-level choreographic programming language HasChor.
%
To evaluate the Async HasChor,

\vspace{0.5cm}
\noindent \textit{\textbf{Contributions.}} The paper makes the following contributions:

\begin{itemize}
\item
    We examine the issues with synchronous communication in choreographic programming and the challenge of making them asynchronous~(\Cref{sec:motiv}).
\item
    We introduce Async HasChor, a choreographic programming language that uses asynchronous communication, empowering programmers to describe distributed protocols previously impossible~(\Cref{sec:tour}).
\item
    We present a formal model of Async HasChor and its endpoint projection that handles out-of-order messages with sequence numbers~(\Cref{sec:formal}).
\item
    Following HasChor, we implement Async HasChor as a library-level choreographic langauge in Haskell using freer monads~(\Cref{sec:formal}).
\item
    We evaluate
\end{itemize}

\noindent We discuss related work in \Cref{sec:related} and conclude in \Cref{sec:conclu}.

\section{Motivation and Challenges}
\label{sec:motiv}

In this section, we examine the issues with synchronous communications in current choreographic programming languages and briefly present our asynchronous extension along with its challenges.

\subsection{Issues with Synchronous Communications}

Existing choreographic programming languages often assume a synchronous communication model~\citep{CC, chor-lambda, pirouette}.
%
The synchronous behavior is evident from the type signature of the communication operator:~\footnote{Again, using pseudo Async HasChor syntax here. Full definition in \Cref{sec:tour}}

\begin{figure}[h]
\begin{cminted}{haskell}
comm :: forall (s :: Loc) (r :: Loc). a @ s -> Choreo (a @ r)
\end{cminted}
\end{figure}

\noindent \texttt{comm} takes a sender and receiver location and moves a value from the sender (\hs{a @ s}) to the receiver (\hs{a @ r}) in the \hs{Choreo} monad.
%
The value is immediately usable at the receiver, indicating the communication has completed.
%
While easy to reason about and simple to implement, synchronous communication significantly restricts the efficiency and expressivity of a choreography:

First, it limits the amount of concurrency a choreography can have since both the sender and receiver are blocked until the communication is completed.
%
For example, the data splitting phase of a MapReduce-like~\citep{map-reduce} system can be represented as the following choreography:

\begin{figure}[h]
\begin{cminted}{haskell}
split bigData = do
  comm @Leader @Worker1 (partition bigData 3)
  comm @Leader @Worker2 (partition bigData 3)
  comm @Leader @Worker3 (partition bigData 3)
\end{cminted}
\end{figure}

\noindent The code clearly expresses the functionality of the application.
%
However, the performance is suboptimal as all the communications are executed sequentially, making the total completion time is the \emph{sum} of all network requests.
%
This becomes particularly problematic when dealing with data in big size.
%
Ideally, we want to exploit concurrency and overlap these communications.

Second, the synchronous assumption makes it impossible to handle message delays and losses, which are very common in distributed systems and crucial for achieving fault tolerance.
%
For example, a distributed key-value store would have a primary server that interacts with the client and a set of replicas to reduce the risk of data loss and improve availability.
%
To update a key-value pair, the primary only needs to receive a \emph{quorum} of acknowledgements, rather than waiting for responses from all replicas.
%
This allows the system to continue functioning even if some replicas fail or parts of the network are unavailable, while still maintaining consistency.

Some extentions...

\subsection{Bring Choreographic Programming into the Future}

\subsection{Get the Right Message}

\section{A Tour of Async HasChor}
\label{sec:tour}

\section{Implementation}
\label{sec:impl}

\section{Formal Model}
\label{sec:formal}

\section{Related Work}

\section{Conclusion}
\label{sec:conclu}


\bibliographystyle{ACM-Reference-Format}
\bibliography{references}

\end{document}
